\chapter{附录}

\section{你应该知道的黑科技}
\paragraph{TNT神教}使用地图编辑器或mod将炸药或地雷放在宝箱下可以无限引爆。
\paragraph{传送器浮空}将家具放在传送机上,然后挖掉传送机下方的支撑方块,家具就可以浮空。有一些例外。
\paragraph{传送枪加速}手持传送枪时水平速度不变且无视上限,竖直速度有加成。经过传送门时速度大小不变。所以可以通过传送门将所有速度转化为水平速度,再利用重力提高竖直加速度从而增加总速度,反复进行这个过程,理论上可以无限加速。但是由于物理帧判定问题,实际操作非常困难,用简单装置只能加速到500mph左右。

\section{高质量电路作品视频}
\subsection{视觉效果}
\begin{itemize}
\item Terraria Bad Apple!! \url{https://www.bilibili.com/video/av46694445}
\item 【泰拉瑞亚】彩虹小电视我们起飞 \url{https://www.bilibili.com/video/av16601896}
\item {[}Terraria]官方推荐电路作品-Bad apple(音乐为后期添加) \url{https://www.bilibili.com/video/av22343683}
\item 【Terraria电路】像素盒定格动画 兔子 \url{https://www.bilibili.com/video/av50343156}
\item {[}Terraria] Digi-Comp 2装置展示-Zerogravitas \url{https://www.bilibili.com/video/av22690346}
\item 泰拉瑞亚 有趣的矿车 \url{https://www.bilibili.com/video/av5271577}
\item 【Terraria电路】给女朋友的生日礼物 \url{https://www.bilibili.com/video/av21009075/}
\item Terraria in-game Pixel Art Animation \url{https://www.bilibili.com/video/av5301176/?p=5}
\item Terraria 莫尔条纹动画 兔子 \url{https://www.bilibili.com/video/av49966963}
\item Terraria 电子钟 \url{https://www.bilibili.com/video/av55613747}
\item Terraria 异世界四重奏 ED『異世界ガールズ・トーク』(传送器定格动画) \url{https://www.bilibili.com/video/av54977071}
\end{itemize}

\subsection{解谜冒险}
\begin{itemize}
\item {[}Terraria]障碍挑战地图-Mappygaming \url{https://www.bilibili.com/video/av22787315}
\item 泰拉瑞亚 terraria 超难创意解谜地图通关介绍 (上) \url{https://www.bilibili.com/video/av19793617}
\item 泰拉瑞亚 terraria 超难创意解谜地图通关介绍 (中) \url{https://www.bilibili.com/video/av20101902}
\item 泰拉瑞亚 terraria 超难创意解谜地图通关介绍 (下) \url{https://www.bilibili.com/video/av20524074}
\item 泰拉瑞亚 在游戏里面玩游戏 \url{https://www.bilibili.com/video/av6594668}
\item {[}Terraria]使用传送枪达到最大速度!3000+mph! \url{https://www.bilibili.com/video/av24580434}
\item【Terraria】【FuryForged】冒险地图Harbinger通关实况 \url{https://www.bilibili.com/video/av37553671/}
\item 在Terraria中玩俄罗斯方块!? \url{https://www.bilibili.com/video/av38924330}
\item 在Terraria中解魔方? (Read Description) \url{https://www.bilibili.com/video/av56760618}
\end{itemize}

\subsection{刷怪刷物品}
\begin{itemize}
\item {[}Terraria]半砖松露蠕虫农场(390+条/小时)!-DicemanX \url{https://www.bilibili.com/video/av23029412}
\item 泰拉瑞亚-诈个尸 \url{https://www.bilibili.com/video/av13411383}
\item 泰拉瑞亚 雕像回血裸装专家月总 \url{https://www.bilibili.com/video/av6393957}
\item 【Terraria电路】【Joe Price】全自动刷怪场、刷BOSS合集 专家模式 \url{https://www.bilibili.com/video/av32865707/}
\item Terraria 1.3.1 Pumpkin Moon Final Wave at 806 pm, 255 platinum coins in a night \url{https://www.bilibili.com/video/av5356226/?p=5}
\item Terraria 1.3 Frost Moon Final Wave at 856 pm, 75820 Total Points - World Record \url{https://www.bilibili.com/video/av5356226/?p=6}
\item 【泰拉瑞亚】最快的boss击杀集锦 无法被超越的世界纪录? \url{https://www.bilibili.com/video/av20725652}
\item 南瓜神教-旧日军团 \url{https://www.bilibili.com/video/av10885401/?p=3}
\item 南瓜神教-霜月夜 \url{https://www.bilibili.com/video/av10885401/?p=4}
\item 【Terraria电路】改进的高效全自动树场 \url{https://www.bilibili.com/video/av38882621/}
\item Terraria 刷石机 \url{https://www.bilibili.com/video/av50203346}
\item Terraria 水晶碎块农场 \url{https://www.bilibili.com/video/av57287619}
\end{itemize}

\section{待实现的电路或装置}

\subsection{电子钟}
原理很简单,使用假人驱动计数器。需要实现功能:精确到游戏秒,显示月相,进地图启动,使用日晷后会校正时间。可考虑的功能:12小时制与24小时制转化,自定义闹钟。

尽管有人已经做了一个电子钟\footnote{\url{https://www.bilibili.com/video/av55613747?from=search&seid=14615669922072114050}},但是它没有实现12小时与24小时转化功能与闹钟功能,也没有精确到游戏秒。

\subsection{确定有限状态自动机}
虽然确定有限自动机的组合逻辑构造与状态转换表有关,但是自动机的整体结构还没有一个优秀的可用方案。主要的难点可能在于接线,因为自动机中的线路是循环的,这违背了泰拉瑞亚中逻辑电路的一般规律。

\subsection{俄罗斯方块}
像素盒显示器的宽度限制是非常讨厌的,这使它无法实现我们一般习惯的横向显示器。另一方面,由于宽度被限制在24格之内,其实际能显示的东西非常有限。俄罗斯方块是为数不多的像素盒显示器可以胜任的任务。经典的俄罗斯方块主显示屏宽10格,高20格。副显示屏用来显示下一个方块,2*4或4*2皆可。使用方向感应器控制。

有人已经做过俄罗斯方块了\footnote{\url{https://www.bilibili.com/video/av38924330}},但是它不能变速,不能计分,操作也不方便。

\subsection{贪吃蛇}
贪吃蛇是像素盒显示器能胜任的另一个任务。其缺点是像素盒显示器是单色的,可能无法分辨身体和果实。ROOM-屠宰场与 TheRedStoneCrafter 已经分别做了一些工作。

\subsection{魔方}
使用彩线灯泡做显示。

有人已经做过魔方了\footnote{\url{https://www.bilibili.com/video/av56760618}},但是它操作极不方便。

\subsection{Flappy Bird}
虽然 TheRedStoneCrafter 已经做过了Flappy Bird\footnote{\url{https://www.bilibili.com/video/av24265449}},但是其效果并不好,还原度可以更高。首先是人物控制,可以利用水+海神贝壳模拟 Flappy Bird 中的运动,用尖刺或木尖刺模拟障碍物,检测人物位置,当人物被击退时游戏结束。

虽然使用递次电路可以实现,但是其占地过大,使得纵向上每个单元至少占三格。使用1秒计时器串联是最节省空间,静态视觉效果最好的,但是其频率过低。

可以用蜂蜜中的飞镖模拟障碍物,但是飞镖在蜂蜜中几乎不可见,可以使用弹幕踏板激活其他光源或虚实化尖刺来提高视觉效果。使用飞镖的好处就是电路难度较低,只需要设计发射飞镖的电路,其余的动画效果都可以自动完成。

\subsection{2048}
使用四向操纵板操作。因为多位数显面积太大,用指数表示,0表示2,1表示4,2表示8,……,9表示1024,A表示2048。如果要超过2048,用b表示4096,c表示8192,d表示16384,E表示32768,F表示65536,G表示131072(可能的最大值)。

使用像素盒显示可以显示出移动动画,但是难度相当大。

\subsection{计算机}

\section{冷却时间机制}
有不少电路物品都有冷却时间。游戏使用一个大小为1000的列表(以下称为冷却列表)存储冷却中的图格及其剩余冷却时间,每帧中,整个列表中的剩余冷却时间减1,如果减到0则将该图格从冷却列表中删除。当有冷却时间的图格被激活,游戏尝试将该图格加入冷却列表,如果该图格已经存在冷却列表中,或者冷却列表已满,则加入失败,该图格激活失败。所以,不光图格正在冷却时无法激活,在整个冷却列表已满时也无法激活。

使用冷却时间机制的图格及其冷却时间如下表:
\begin{table}[!h]
\centering
\begin{tabular}{c|c}
图格名称&冷却时间/帧\\\hline\hline
炮台&30\\\hline
雪球发射器&10\\\hline
烟花盒&30\\\hline
烟花喷泉&30\\\hline
矿车轨道&5\\\hline
飞镖机关&200\\\hline
超级飞镖机关&200\\\hline
烈焰机关&200\\\hline
长矛机关&90\\\hline
尖球机关&300\\\hline
喷泉(机关)&200\\\hline
刷怪雕像&30\\\hline
刷物品雕像&600\\\hline
传送NPC雕像&300\\\hline
\end{tabular}
\end{table}

此外,还有一些图格没有冷却时间,但是也借用了冷却列表用来做倒计时,包括计时器和引爆器。

\section{机关射弹的生成和刷新机制}
射弹生成中的参数包括射弹坐标、射弹速度、射弹ID、伤害、击退,和其他所需参数。未经说明的情况下,长度单位为像素,时间单位为帧。例如,速度的单位为像素/帧。

游戏中活跃的射弹被存储在一个大小为1000的列表中。新生成射弹时,会将新的射弹放入列表中第一个空位;如果列表已满,则射弹不会生成(但是不代表与射弹生成并列的其他操作也会中止)。

射弹是通过每帧中的刷新操作前进的。所谓刷新操作,就是根据射弹当前的属性计算出射弹下一帧的属性并更新。例如,已知射弹当前坐标和射弹速度,则射弹下一帧的坐标是当前坐标+速度。不同的射弹在更新时做出的判断和操作各不相同,甚至某些射弹在一帧中会刷新多次。非穿墙的射弹每次刷新时,如果与某个青绿压力垫板碰撞,并且在刷新前未与该青绿压力垫板碰撞,那么会激活该青绿压力垫板。

射弹的销毁操作会将列表中该射弹位置清空。销毁操作发生的场合主要有:不穿墙的射弹击中实体块;射弹存在时间达到了生存期;射弹离开了地图;有限穿透的射弹达到了穿透上限。

之所以在电路教程里说这么详细,是因为有一些奇怪的情况需要解释。

首先需要注意的一点是,射弹激活青绿压力垫板只发生在刷新后,而不发生在生成后。而刷新时激活青绿压力垫板的条件包括“刷新前未碰撞”。这意味着如果一个射弹在生成时与青绿压力垫板碰撞,那么既不在生成后激活,又不满足刷新时的激活条件,则这个青绿压力垫板不会被激活。

一个更极端的例子是飞镖机关-青绿压力垫板的连锁反应。

\subsection{匀速直线运动的射弹及其速度大小}
飞镖机关的飞镖速度12,超级飞镖机关的飞镖速度12,彩纸炮的引导速度14,传送枪台的传送枪子弹速度93。

\subsection{抛射型射弹的运动机制}
电路产生的抛射型射弹有大炮和兔兔炮的炮弹、雪球发射器的雪球。这类抛射型射弹在生成后首先匀速直线运动一段时间(大炮和兔兔炮为17,雪球发射器为19),然后获得一个垂直方向的加速度(大炮和兔兔炮为0.28,雪球发射器为0.3),并且水平速度依指数衰减(大炮和兔兔炮为0.99,雪球发射器为0.98)。

\subsection{炮台发射的射弹的初速度方向}
炮台一共有9个朝向,其对应的速度方向如\autoref{}所示。

\subsection{射弹碰撞箱及初始生成位置}

\subsection{实验测定抛射射弹轨迹}