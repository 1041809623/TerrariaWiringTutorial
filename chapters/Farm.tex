\chapter{农场}

前面我们主要围绕电路理论来讲。这一章中我们会看到一些利用游戏机制的精妙设计。

\section{水晶碎块农场}
\subsection{水晶碎块的生成机制}
在\textbf{困难模式}中,游戏每一帧会执行如下的步骤:
\begin{enumerate}
    \item \textbf{选取支撑块。}在一个超大的矩形区域内随机取一格,如果恰好是粉冰雪块或珍珠石块,并且取的这一格在洞穴或地狱,那么有1/110的概率可以进入下一步,否则不生成水晶。这个矩形区域左右端距离世界左右端各10格,上端为地表,下端距离世界下端20格。
    \item \textbf{选取生成格。}把上一步中取到的一格叫做支撑块。在支撑块下左右三个相邻格中随机取一格,其中取下的概率是1/2,左右的概率各1/4,这一格就作为水晶的生成格。生成格不能被其他图格阻挡,否则不能生成水晶。
    \item \textbf{局部数量限制。}统计以支撑块为中心,13格$\times$13格的正方形区域内已经存在的水晶碎块数量,如果超过了1,那么不能生成水晶。
    \item \textbf{生成水晶。}如果前三步所有检查均通过,那么在生成格生成水晶。
\end{enumerate}

\subsection{三大设计要点}
一个水晶农场主要分三个部分:生长、收割、收集。在生长环节,需要选取使用粉冰雪块还是珍珠石块,还需要设计方块的摆放方式。收割可以通过简单的虚化实现,但是我们需要知道收割的频率。如果使用相对高频的驱动进行收割,那么比较消耗CPU资源;如果使用过于低频的驱动进行收割,那么容易达到局部上限。在收集环节,需要选取使用半砖、传送带还是传送机。半砖速度快,稳定性差;传送带速度慢,稳定性好;传送机速度快,稳定性快,但是不方便操作。所有这三个部分的方案设计都是通过计算实现的。

要生成水晶碎块,需要通过四层检查:选取的方块恰好可以做支撑块;1/110的概率;生成格不能被阻挡;局部数量限制。其中1/110的概率是固定值,先不管。局部数量限制是与基础生成速度相关的,如果水晶碎块生成得快,那么就很容易达到局部数量上限,否则可以忽略掉局部数量上限。

选取方块恰好可以做支撑块的概率,是矩形区域内,洞穴和地狱中粉冰雪块和珍珠石块的比例。我们先假设这个比例是100\%。生成格不能被阻挡,意味着支撑块不能过于密集,不过我们先不管,先假设这个概率也是100\%。这样的话,水晶碎块的最大生成速率是1个/110帧,这是整个世界中水晶碎块的速率。如果要知道13$\times$13区域的生成速率,就要用这个区域的面积除以大的矩形区域面积。换句话说,一个固定大小区域内,水晶碎块的生成速率与选取支撑块的矩形区域的面积成反比。这个矩形区域的面积差不多等于世界的宽度$\times$地下洞穴地狱的总高度,所以地图越小,局部的生成速率越快;地下洞穴地狱的总高度越小,局部生成速率越快。

再注意一点,矩形区域包括了地下、洞穴、地狱,而支撑块只能在洞穴、地狱,这意味着地下层的高度在地下洞穴地狱的总高度中占比越小,局部生成速率越快。

如果游戏中的一天时间,在13$\times$13区域内平均只能生成一个水晶,那么我们可以认为这个生成速率在每天收割一次的情况下不会受到局部生成上限的限制。已知整个世界中水晶生成速率是1个/110帧,游戏中一天是86400帧,那么换算过来游戏中一天可以生成785个水晶。在13$\times$13=169格区域内只有一个水晶,那么矩形区域的面积为785$\times$169=132665格。小世界的宽度是4200格,所以小世界的矩形区域宽度为4180格,换算过来,矩形区域高度只有32格。因为矩形区域包括地下、洞穴、地狱,所以我们估计它的高度至少有640格,为32格的20倍。所以小世界中,平均要经过20天,才能让13$\times$13区域内水晶数量的期望值达到1。实际应用中,我们不需要考虑水晶的局部上限,因为达不到。至于收割频率,使用逻辑传感器(昼)或(夜),一天一次就够了。

另一个上限是掉落物上限。水晶收割后,如果收集过慢,可能因为达到掉落物上限而消失。整个世界中水晶生成为1个/110帧,达到400的掉落物上限需要400$\times$110=44000帧,超过了12分钟,所以只要水晶收割后12分钟内可以收集起来,就不会达到掉落物上限,因此横向的运输使用传送带就够了,没有必要用半砖。纵向运输半砖占用体积更小,可以选择半砖。

最后我们来看生长环节。方块肯定选择珍珠石块,因为粉冰雪块下面会长冰锥。设计石块排列方式的时候,需要记住我们之前的结论,认为水晶不会达到局部上限。为了尽可能快的生成水晶,我们需要增加石块密度的同时,增加每个石块的暴露面积,所以石块排列选取\autoref{}的方案。这个方案中,多个支撑块共用了生成格,所以有可能导致生成水晶被之前的水晶阻挡。但是我们之前已经说过了,连局部上限都达不到,阻挡的情况更难达到,所以可以忽略。

\section{全自动树场}

\section{生命果/世花农场}

\section{液体反应池}

\section{松露虫农场}

\section{采沙场}

\section{天梯神教}
\subsection{南瓜月天梯神教}
\subsection{霜月天梯神教}

\section{南瓜神教}

\section{全自动月亮事件}

\section{DPS纪录}

\section{速度纪录}

\section{Boss速杀场地}
