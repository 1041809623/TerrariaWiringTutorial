\chapter{农场}

前面我们主要围绕电路理论来讲。这一章中我们会看到一些利用游戏机制的精妙设计。

\section{水晶碎块农场}
\subsection{水晶碎块的生成机制}
在\textbf{困难模式}中,游戏每一帧会将如下的步骤执行多次,执行次数=世界宽度$\times$世界高度$\times$0.000015:
\begin{enumerate}
    \item \textbf{选取支撑块。}在一个超大的矩形区域内随机取一格,如果恰好是粉冰雪块或珍珠石块,并且取的这一格在洞穴或地狱,那么有1/110的概率可以进入下一步,否则不生成水晶。这个矩形区域左右端距离世界左右端各10格,上端为地表,下端距离世界下端20格。
    \item \textbf{选取生成格。}把上一步中取到的一格叫做支撑块。在支撑块下左右三个相邻格中随机取一格,其中取下的概率是1/2,左右的概率各1/4,这一格就作为水晶的生成格。生成格不能被其他图格阻挡,否则不能生成水晶。
    \item \textbf{局部数量限制。}统计以支撑块为中心,13格$\times$13格的正方形区域内已经存在的水晶碎块数量,如果超过了1,那么不能生成水晶。
    \item \textbf{生成水晶。}如果前三步所有检查均通过,那么在生成格生成水晶。
\end{enumerate}

\subsection{三大设计要点}
一个水晶农场主要分三个部分:生长、收割、收集。在生长环节,需要选取使用粉冰雪块还是珍珠石块,还需要设计方块的摆放方式。收割可以通过简单的虚化实现,但是我们需要知道收割的频率。如果使用相对高频的驱动进行收割,那么比较消耗CPU资源;如果使用过于低频的驱动进行收割,那么容易达到局部上限。在收集环节,需要选取使用半砖、传送带还是传送机。半砖速度快,稳定性差;传送带速度慢,稳定性好;传送机速度快,稳定性快,但是不方便操作。所有这三个部分的方案设计都是通过计算实现的。

要生成水晶碎块,需要通过四层检查:选取的方块恰好可以做支撑块;1/110的概率;生成格不能被阻挡;局部数量限制。其中1/110的概率是固定值,先不管。局部数量限制是与基础生成速度相关的,如果水晶碎块生成得快,那么就很容易达到局部数量上限,否则可以忽略掉局部数量上限。

选取方块恰好可以做支撑块的概率,是矩形区域内,洞穴和地狱中粉冰雪块和珍珠石块的比例。我们先假设这个比例是100\%。生成格不能被阻挡,意味着支撑块不能过于密集,不过我们先不管,先假设这个概率也是100\%。这样的话,水晶碎块的最大生成速率是(世界宽度$\times$世界高度$\times$0.000015)个/110帧,这是整个世界中水晶碎块的速率。那么每格上水晶碎块的速率就是0.000015个/110帧\footnote{与worldSurface和rockLayer占世界高度比例相关,这里是非常粗略的估算}。13$\times$13区域的生成速率就是$0.000015\times 13^2$=0.002535个/110帧。达到局部生成上限平均需要2个/(0.002535个/110帧)=86785帧,大约是1天。

回到前两个概率。前两个概率合起来,就是任选一个方块的一个表面,可以生成水晶的概率。换句话说,生成速率与可生成的表面积成正比。\autoref{}所示的排列方式可以达到最大的表面积,在这个排列下,任选一个方块,可以做支撑块的概率是1/2,生成格不被阻挡的概率是100\%,所以水晶碎块的生成速率全部要除以2,达到局部生成上限平均需要约2天。收割周期确定为1天左右,这样不会过于频繁,也不会有太多方块达到上限。

然后考虑收割方式,是一次性虚化收割,还是一块一块收割?掉落物上限是400,一天收割一次的话,要求水晶农场的面积控制在$13\times 13\times 400\textrm{格}=67600$格以内。如果规模更大,就需要一块一块收割。这里每块的大小设置取决于收割后运输的速度。假设总共有$N$块,那么每块的收割间隔是$(86400/N)$帧。假设总面积是$S$格,那么每块的面积是$(S/N)$格,每块收割时掉落约$(S/N/169)$个水晶碎块。达到400个水晶碎块,需要收割$400/(S/N/169)=(67600N/S)$块,也就是说,水晶从掉落到收集,时间是$(86400/N)\times(67600N/S)=(5840640000/S)$帧,超过这个时间就很有可能因为掉落物上限损失水晶。对于大世界来说,$S$至多是$8400\times2400=20160000$格,时间大约是290帧,不到5秒。在5秒内收集全世界的水晶是不可能的。

水晶农场面积越小,对收集速度的要求就越低。当规模较小时,可以使用\autoref{}最大化生成速率;当规模较大时,需要优先考虑便于半砖与传送器运输的结构以增加收集速度。

\section{全自动树场}
\subsection{树的生成机制}
\begin{note}
本小节内容来自\href{https://github.com/sgkoishi}{sgkoishi}在\href{https://github.com/putianyi889/TerrariaWiringTutorial/pull/16}{项目PR}中的贡献。
\end{note}
本小节中,“玩家附近”指玩家中心的左右0.6sWidth(默认72格),上下0.6sHeight(默认40.5格)范围内。\footnote{相关代码位于\path{bool Terraria.WorldGen.PlayerLOS(int, int)}。}

树木生长只会发生在地图中的矩形区域内。\footnote{相关代码位于\path{void Terraria.WorldGen.UpdateWorld()}。}

地表树类的矩形区域左右端和上端均距离地图边缘10格,下端为\path{worldSurface}。

每次更新中,重复以下步骤($\textnormal{世界宽度}\times\textnormal{世界高度}\times 0.00003$)次:
\begin{itemize}
\item 随机选取地表矩形区域内的一个图格。
\item 如果该图格为橡树种(ID 20):
	\begin{itemize}
	\item 该格内液体量不超过32(约八分之一)。
	\item 位于玩家附近'
	\item 如果满足了本步骤中所有条件,则有1/20的几率长出树木。游戏会根据该图格的位置向下寻找最高的非橡树种方块,并将其标记为树根下方的方块。
	\end{itemize}
\item 如果该图格为丛林草地(ID 60):
	\begin{itemize}
	\item 有1/7的几率检测该图格上方有没有方块,如果没有,则长出丛林草。
	\item 如果符合以下条件,则有1/500的几率长出树木(该格视为树根下方的方块):
		\begin{itemize}
		\item 上一步中没有触发检测。
		\item 该图格上方没有方块,或只有丛林草/刺。
		\item 位于玩家附近。
		\end{itemize}
	\end{itemize}
\item 如果该格内方块类型为蘑菇草(ID 70):
	\begin{itemize}
	\item 有1/10的几率长出小蘑菇。
	\item 如果位于玩家附近,则有1 / 100的几率长出地表蘑菇树。
		\begin{itemize}
		\item 该格视为树根下方的方块。
		\end{itemize}
	\end{itemize}
\end{itemize}
树根下的方块上方相邻的方块标记为树根。

地下树类的矩形区域左右端均距离地图边缘10格,上端为\path{worldSurface},下端为距离地图边缘20格。

每次更新中,重复以下步骤($\textnormal{世界宽度}\times\textnormal{世界高度}\times 0.000015$)次:
\begin{itemize}
\item 随机选取地下矩形区域内的一个图格。
\item 如果该格内方块类型为蘑菇草(ID 70):
	\begin{itemize}
	\item 有1/10的几率长出小蘑菇。
	\item 如果位于玩家附近,则有1/200的几率长出地下蘑菇树。
	\end{itemize}
\end{itemize}

以上提到的所有“长出树木”的过程,分为棕榈木、地下蘑菇树、其他树木。

\paragraph*{其他树木\footnote{相关代码位于\path{bool Terraria.WorldGen.GrowTree(int, int)}。}}
包括普通、丛林、腐化、猩红、神圣、地表蘑菇、苔原七种树。丛林木高度上限为21,其余树木高度上限为16。

\begin{itemize}
\item 树根下方方块长有对应类型的草皮,或是雪块。
\item 树根下方的图格,其左右相邻的方块之一也符合该条件。
\item 树根所在图格完全没有液体,除非树根下的方块为丛林草地。
\item 树根所在格的背景墙为栅栏(各种木栅栏或金属栅栏),或无背景墙。
\item 树根左右2格、上方高度上限格(包括树根)的矩形区域内没有方块。
\end{itemize}
符合上述条件后,将会生成一棵树,高度为5-高度上限中的随机数值。

从树根往上,每个图格都有几率出现树枝。只有左边有、只有右边有和左右都有树枝的几率均为1/10。同一边不会出现两个上下相邻的树枝。

树根下方左右相邻的图格里,符合第一点(长草或雪地)的方块上将会长出另一块树根。

\paragraph*{棕榈木\footnote{相关代码位于\path{bool Terraria.WorldGen.GrowPalmTree(int, int)}。}}
\begin{itemize}
\item 树根下方方块为沙块(普通、腐化、神圣、猩红之一)。
\item 树根所在图格没有任何液体,也没有任何背景墙。
\item 树根左右1格、上方30格(包括树根)的矩形区域内没有方块。
\end{itemize}
符合上述条件后,将会生成一棵树,高度为10-21中的随机数值。

\paragraph*{地下蘑菇树\footnote{相关代码位于\path{bool Terraria.WorldGen.GrowShroom(int, int)}。}}
\begin{itemize}
\item 树根左右两格没有岩浆。
\item 树根下方左右相邻的方块均为蘑菇草地。
\item 树根左右2格,上方13格(包括树根)的矩形区域内没有方块,自然生长的蘑菇(ID 71)除外。
\end{itemize}
符合上述条件后,将会生成一棵树,高度为4-11中的随机数值。

\section{生命果/世花农场}
1

\section{液体反应池}
1

\section{松露虫农场}
1

\section{采沙场}
1

\section{彩虹砖农场}
彩虹砖农场的核心思路是最大化彩虹史莱姆的生成率。彩虹史莱姆在\nameref{app31}中出现在\nameref{app32},优先级较低。我们需要在保留彩虹史莱姆刷怪条件的前提下尽可能阻止优先级高于彩虹史莱姆的刷怪。彩虹史莱姆刷怪要求刷怪面不低于worldSurface,神圣环境,困难模式,cloudAlpha大于0。
\begin{longtable}{|c|c|}
\hline
刷怪种类&阻止方法\\\hline
\endfirsthead
\hline
刷怪种类&阻止方法\\\hline
\endhead
\hline
\endfoot
\makecell{四柱/事件/蜘蛛巢/地下沙漠/地牢/陨石/\\ 撒旦军队/霜月/南瓜月/日食/腐地蠕虫/\\ 地下蠕虫/老鼠/蜗牛/沙尘暴/附魔剑}&\makecell{容易排除,或者\\ 已经被彩虹史莱姆\\ 刷怪要求排除}\\\hline
太空&刷怪场远离太空\\\hline
昏迷男子/沉睡渔夫/受缚哥布林/受缚巫师&所有城镇NPC入住\\\hline
\makecell{巨骨舌鱼/血水母/嗜血怪/海洋/\\ 食人鱼/蓝水母/水中小动物}&避免水中刷怪\\\hline
小动物&减少生成小动物概率\\\hline
发光蘑菇地&刷怪面不能是蘑菇草皮\\\hline
宝箱怪&清除刷怪面上方的天然土墙\\\hline
幻灵/弹跳杰克南瓜灯&避开晚上\\\hline
骷髅博士/紫胶虫/困难模式丛林/丛林&刷怪面不能是丛林草皮\\\hline
丛林青蛙&避开丛林环境\\\hline
木乃伊&刷怪面不能是各种沙块\\\hline
地表神圣&\makecell{刷怪面不能是珍珠沙块、\\ 珍珠石块、神圣草皮、粉冰雪块}\\\hline
猩红之地&\makecell{刷怪面不能是猩红石块、猩红沙块、\\ 红冰雪块、猩红草皮、猩红矿}\\\hline
腐化之地&\makecell{刷怪面不能是黑檀石块、黑檀沙块、\\ 紫冰雪块、腐化草皮、魔矿}\\\hline
冰雪巨人&避开苔原环境
\end{longtable}
所以我们可以选择传送带做刷怪面,方便收集彩虹砖的同时也可以满足刷怪面的要求。

接下来尝试优化\nameref{app33}。困难模式中刷怪率为540,刷怪量为6。我们使用地表白天,所以跳过了第2、5、8步。我们规避了苔原环境所以跳过了第3步。第4步中,我们规避了地牢、沙尘暴、地下沙漠、丛林、陨石;神圣环境与邪恶环境不能共存,所以第4步也跳过了。第6步自然跳过。第7步暂时不考虑。使用水蜡烛和战斗药水,执行第9步得到刷怪率为202,刷怪量为18\footnote{要记住我们已经规避了太空,所以没有最后一项的计算}。

这个结果距离刷怪率极限还差很远,所以我们还要考虑增加一些环境,在可能覆盖彩虹史莱姆的刷怪和刷怪率中做取舍。

如果使用丛林环境,那么彩虹史莱姆有1/9会被青蛙取代,也就是说彩虹史莱姆的刷怪速度变为了8/9。增加了丛林环境后刷怪率为81。等效的彩虹史莱姆的刷怪率为$81/8\times 9=91.125$,好了不少。再考虑刷怪率计算的第7步,到这一步时刷怪率为216,刷怪量为9。如果可以将活跃敌怪数控制在最多3个,那么考虑到第7步,最终的刷怪率就可以到56,超过了60的极限。

综上所述,再考虑到彩虹史莱姆的生成上限(1个),我们的彩虹砖农场要求如下:
\begin{itemize}
\item 刷怪面使用传送带。
\item 使用神圣丛林环境、地表层。整个刷怪区域和刷怪面需要回避地下、太空、海洋、苔原、沙漠。
\item 需要在雨天。尽量避开晚上,如果在晚上\footnote{尤其是新月},幻灵会占用掉部分刷怪机会。
\item 使用水蜡烛和战斗药水以增加刷怪速度。
\item 需要尽快清掉刷出来的怪,整个刷怪场存活的敌怪需要控制在至多3个,以达到最大刷怪速度。
\item 远离所有城镇NPC,以阻止小动物刷怪。
\end{itemize}

\section{天梯神教}
1
\subsection{南瓜月天梯神教}
1
\subsection{霜月天梯神教}
1

\section{南瓜神教}
1

\section{全自动月亮事件}
1

\section{DPS纪录}
1

\section{速度纪录}
1

\section{Boss速杀场地}
1
