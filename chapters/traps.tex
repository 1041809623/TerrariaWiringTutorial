\chapter{机关射弹的生成和刷新机制}
射弹生成中的参数包括射弹坐标、射弹速度、射弹ID、伤害、击退,和其他所需参数。未经说明的情况下,长度单位为像素,时间单位为帧。例如,速度的单位为像素/帧。

游戏中活跃的射弹被存储在一个大小为1000的列表中。新生成射弹时,会将新的射弹放入列表中第一个空位;如果列表已满,则射弹不会生成(但是不代表与射弹生成并列的其他操作也会中止)。

射弹是通过每帧中的刷新操作前进的。所谓刷新操作,就是根据射弹当前的属性计算出射弹下一帧的属性并更新。例如,已知射弹当前坐标和射弹速度,则射弹下一帧的坐标是当前坐标+速度。不同的射弹在更新时做出的判断和操作各不相同,甚至某些射弹在一帧中会刷新多次。非穿墙的射弹每次刷新时,如果与某个青绿压力垫板碰撞,并且在刷新前未与该青绿压力垫板碰撞,那么会激活该青绿压力垫板。

射弹的销毁操作会将列表中该射弹位置清空。销毁操作发生的场合主要有:不穿墙的射弹击中实体块;射弹存在时间达到了生存期;射弹离开了地图;有限穿透的射弹达到了穿透上限。

之所以在电路教程里说这么详细,是因为有一些奇怪的情况需要解释。

首先需要注意的一点是,射弹激活青绿压力垫板只发生在刷新后,而不发生在生成后。而刷新时激活青绿压力垫板的条件包括“刷新前未碰撞”。这意味着如果一个射弹在生成时与青绿压力垫板碰撞,那么既不在生成后激活,又不满足刷新时的激活条件,则这个青绿压力垫板不会被激活。

一个更极端的例子是飞镖机关-青绿压力垫板的连锁反应。

\section{匀速直线运动的射弹及其速度大小}
飞镖机关的飞镖速度12,超级飞镖机关的飞镖速度12,彩纸炮的引导速度14,传送枪台的传送枪子弹速度93。

\section{抛射型射弹的运动机制}
电路产生的抛射型射弹有大炮和兔兔炮的炮弹、雪球发射器的雪球。这类抛射型射弹在生成后首先匀速直线运动一段时间(大炮和兔兔炮为17,雪球发射器为19),然后获得一个垂直方向的加速度(大炮和兔兔炮为0.28,雪球发射器为0.3),并且水平速度依指数衰减(大炮和兔兔炮为0.99,雪球发射器为0.98)。

\section{射弹碰撞箱及初始生成位置}

\section{实验测定抛射射弹轨迹}