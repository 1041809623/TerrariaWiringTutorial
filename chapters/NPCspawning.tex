\chapter{刷怪机制}
执行刷怪的函数为 NPC.SpawnNPC。

如果NPC.noSpawnCycle为true,那么这一帧不刷怪,把NPC.noSpawnCycle重置为false。

如果要刷怪的话,会对每个未死亡的玩家执行刷怪过程。在史莱姆雨进行时,会在其他刷怪之前插入史莱姆雨的刷怪。史莱姆雨的刷怪是额外刷怪,不会影响正常刷怪。

进行正常刷怪时,首先要判断刷怪类型、刷怪率和刷怪量,然后决定是否要刷怪,然后决定刷怪点和刷怪面,最后决定刷什么怪。

\section{史莱姆雨刷怪}

\section{刷怪类型、刷怪率和刷怪量}

\section{刷怪点和刷怪面}

\section{刷怪种类}
刷怪的优先级是四柱>太空>事件>昏迷男子>蜘蛛洞>地下沙漠>巨骨舌鱼>血水母>嗜血怪>海洋>沉睡渔夫>食人鱼>蓝水母>水中小动物>受缚哥布林>受缚巫师>小动物>地牢>陨石>撒旦军团>霜月>南瓜月>日食>发光蘑菇地。

海洋刷怪中包含部分沉睡渔夫。琵琶鱼包含在食人鱼中。绿水母包含在蓝水母中。

\subsection{四柱}
如果玩家在四柱附近(玩家中心到四柱中心的直线距离不大于4000像素),那么就会刷四柱怪。四柱刷怪的优先级是星云柱>星旋柱>星尘柱>日曜柱。

\subsubsection{星云柱}
刷怪为星云浮怪(ID:420)、吮脑怪(ID:421)、进化兽(ID:423)、预言帝(ID:424)。这四个怪的刷怪比例为1:5:3:3。星云浮怪在整个世界中的上限为2个,进化兽在整个世界中的上限为3个,预言帝在整个世界中的上限为2个。这个刷怪比例\&上限的规则是四柱的特色。举例来说,没有任何刷怪的时候,这四个怪的刷怪概率分别是1/12、5/12、1/4、1/4;如果已经刷出了两个星云浮怪,那么不会再刷星云浮怪,剩下三个怪的刷怪概率分别是5/11、3/11、3/11。

\subsubsection{星旋柱}
刷怪为漩泥怪(ID:425)、异星蜂王(ID:426)、异星黄蜂(ID:427)、星旋怪(ID:429)。刷怪比例为2:1:2:4。漩泥怪上限为3,异星蜂王上限为3,星旋怪上限为4。

\subsubsection{星尘柱}
刷怪为银河织妖(ID:402)、星细胞(ID:405)、流体入侵怪(ID:407)、闪耀炮手(ID:409)、观星怪(ID:411)。刷怪比例为1:1:1:2:3。

\subsubsection{日曜柱}
刷怪为千足蜈蚣(ID:412)、火龙怪(ID:415)、火龙怪骑士(ID:416)、火滚怪(ID:417)、流星火怪(ID:418)、火月怪(ID:419)、火龙战士(ID:518)。刷怪比例为1:1:1:1:1:1:1。千足蜈蚣上限为1,火龙怪上限为2,火龙怪骑士上限为1,火龙战士上限为2。

\subsection{太空刷怪}
优先级是火星飞船>火星探测器>飞龙>鸟妖。

如果执行的是火星入侵的刷怪,那么生成火星飞船(ID:388)。

在石巨人后,并且刷怪点到世界中心的横坐标距离大于世界宽度$\times$0.165\footnote{小世界为693格,中世界为1056格,大世界为1386格},那么有概率生成火星探测器(ID:399),这个概率与是否打过火星入侵、是否在水蜡烛区域、是否有水蜡烛buff相关(\autoref{tab5651})。水蜡烛区域和水蜡烛buff不是一回事,水蜡烛区域不包括手持水蜡烛的情况。火星探测器的上限为1。

\begin{table}[!h]
    \centering
    \begin{tabular}{cccccc}
         000&001&011&100&101&111\\\hline
         1/8&15/64&5/9&1/30&59/900&19/100 
    \end{tabular}
    \caption{二进制的第一位表示是否打过火星入侵,第二位表示是否在水蜡烛区域,第三位表示是否有水蜡烛buff。例如101表示打过火星入侵,不在水蜡烛区域内,有水蜡烛buff。}
    \label{tab5651}
\end{table}

没有水蜡烛buff的时候,飞龙(ID:87)的生成概率为1/10;有水蜡烛buff的时候生成概率为19/100。飞龙的上限为1。当玩家中心在人工背景墙前时不会刷飞龙。

如果火星飞船、火星探测器、飞龙均未生成,那么生成鸟妖。

\subsection{事件刷怪}
\subsubsection{哥布林入侵}
1/9概率生成哥布林巫士(ID:29),8/45概率生成哥布林苦力(ID:26),32/135概率生成哥布林弓箭手(ID:111),32/405概率生成哥布林盗贼(ID:27),64/405概率生成哥布林战士(ID:28)。在困难模式中,有1/30概率生成哥布林召唤师(ID:471),这会覆盖前面的生成。哥布林召唤师上限为1。

\subsubsection{雪人入侵}
1/7概率生成巴拉雪人(ID:145),2/7概率生成雪人暴徒(ID:143);4/7概率生成戳刺先生(ID:144)。

\subsubsection{海盗入侵}
1/11概率生成海盗弩手(ID:215),10/99概率生成鹦鹉(ID:252),80/693概率生成海盗神射手(ID:214),160/693概率生成私船海盗(ID:213),320/693概率生成海盗水手(ID:212)。

海盗船长(ID:216)有1/30概率生成,会覆盖前面的生成。海盗船长上限为1。

海盗船(ID:491)生成要求入侵进度超过一半,并且刷怪面的左右各20格,上方10格到40格范围内没有实体块。海盗船生成概率是1/20。海盗船上限为1。海盗船的生成会覆盖其他生成。

\subsubsection{火星入侵}
火星飞碟(ID:395)的生成分为两段判定。离入侵结束不到100分\footnote{入侵事件总分为160+40$\times$玩家数量}时,火星飞碟的概率为1/10并且会覆盖其他生成(包括第二段)。第二段中火星飞碟和其他敌怪处理方法相同。

火星飞碟上限为1,火星走妖(ID:520)上限为1。以下是第二段判定。

火星飞碟概率为1/70,鳞甲怪枪手(ID:390)和火星工程师(ID:386)概率均为9/140(火星飞碟达到上限的话,这个概率变为1/14),火星飞船(ID:388)概率为2/35,扰脑怪(ID:381)概率为4/35,激光枪手(ID:382)概率为4/35,火星走妖(ID:520)概率为1/7,灰咕噜兽(ID:385)、电击怪(ID:389)和火星军官(ID:383)概率均为1/7(火星走妖达到上限的话,这个概率为4/21)。

\subsection{昏迷男子}
满足昏迷男子生成条件,并且不是水中刷怪,那么有1/80概率生成昏迷男子。

\subsection{蜘蛛洞}
刷怪面有蜘蛛墙,不高于 rockLayer,距离世界底端大于210格,且不是水中刷怪,未解救过发型师,那么有1/8概率生成受缚发型师(ID:354)。

在未生成受缚发型师的前提下,如果刷怪面有蜘蛛墙或者判定为蜘蛛洞刷怪,那么困难模式生成黑隐士(ID:163),困难模式前生成爬墙蜘蛛(ID:164)。

\subsection{地下沙漠}
进行地下沙漠刷怪的要求是刷怪面在地下,并且刷怪面上的背景墙是硬化沙墙/沙岩墙,或者它们的转化墙\footnote{神圣化、腐化、血腥化}。刷怪优先级是沙虫>墓穴爬虫>困难模式>其他。

沙虫(ID:510)和墓穴爬虫(ID:513)的生成都要求玩家中心不能在人工背景墙前,并且刷怪面在地表下100格以下。沙虫的概率为1/33,墓穴爬虫的概率为1/22。沙虫只会在困难模式生成。

在困难模式中,有4/5的概率进行困难模式刷怪。困难模式的刷怪有腐恶食尸鬼(ID:525)、红染食尸鬼(ID:526)、神梦食尸鬼(ID:527)、食尸鬼(ID:524)、沙漠幽魂(ID:533)、邪恶拉弥亚(ID:529)、沙贼(ID:530)、拉弥亚(ID:528)、蛇蜥怪(ID:532),它们的刷怪比例为2:2:2:2:1:1:1:1:1,腐恶食尸鬼在腐化环境生成,红染食尸鬼在血腥环境生成,神梦食尸鬼在神圣环境生成,食尸鬼只在纯净环境生成,沙漠幽魂和邪恶拉弥亚在腐化或血腥环境生成,沙贼和拉弥亚在没有腐化和血腥的环境生成,蛇蜥怪不挑环境。举例来说,如果玩家同时处在血腥和神圣环境中,那么可以生成红染食尸鬼、神梦食尸鬼、沙漠幽魂、邪恶拉弥亚、蛇蜥怪,其刷怪比例为2:2:1:1:1。

既没刷出沙虫或墓穴爬虫,也没有困难模式刷怪,那么刷蚁狮(ID:69)/蚁狮马(ID:508)/蚁狮蜂(ID:509),刷怪比例为1:3:1。

\subsection{巨骨舌鱼}
困难模式+水中刷怪+丛林环境,2/3概率生成巨骨舌鱼(ID:157)。

\subsection{血水母}
困难模式+水中刷怪+血腥环境,2/3概率生成血水母(ID:242)。

\subsection{嗜血怪}
困难模式+水中刷怪+血腥环境,2/3概率生成嗜血怪(ID:241)。

\subsection{海洋}
要求:水中刷怪,刷怪横坐标距离地图左右边界<250格,刷怪面是沙块或珍珠/黑檀/猩红沙块,刷怪面纵坐标在rockLayer之上。优先级:沉睡渔夫>其他。

\subsubsection{沉睡渔夫}
要求:刷怪横坐标在安全区域外,满足沉睡渔夫生成条件。

从刷怪面向上搜索47格(不包括刷怪面)找到第一个可以生成沉睡渔夫的图格,这一格需要满足:没有实体块;上方第一格没有实体块;上方第二格没有液体;上方第二格没有实体块。如果找到了这一格,那么生成沉睡渔夫(ID:376)。

\subsubsection{其他}
1/60概率生成海蜗牛(ID:220),59/1500概率生成乌贼(ID:221),59/500概率生成鲨鱼(ID:65),413/1500概率生成螃蟹(ID:67),413/750概率生成粉水母(ID:64)。

\subsection{沉睡渔夫}
要求:不是水中刷怪,刷怪横坐标距离地图左右边界<340格,刷怪面是沙块或珍珠/黑檀/猩红沙块刷怪面纵坐标在worldSurface之上,满足沉睡渔夫生成条件。生成沉睡渔夫。

\subsection{食人鱼}
要求:水中刷怪。刷怪面在rockLayer之下,有1/2概率生成食人鱼(ID:58);刷怪面是丛林草皮,必然生成食人鱼。在困难模式,生成的食人鱼有2/3概率转化为琵琶鱼(ID:102)。

\subsection{蓝水母}
要求:水中刷怪,刷怪面在worldSurface之下。有1/3概率生成蓝水母(ID:63)。困难模式中生成的蓝水母转化为绿水母(ID:103)。

\subsection{水中小动物}
如果是水中刷怪,在腐化环境有1/4概率生成腐化金鱼(ID:57)。

如果是水中刷怪,不在腐化环境,刷怪面在worldSurface之上,刷怪面距离世界顶端大于50格,在白天,有1/6概率在水面\footnote{这里的水面判定与海洋沉睡渔夫的判定相同。}等概率生成鸭(ID:364)或野鸭(ID:362)。上一句话中没有成功生成的话,生成金鱼(ID:55)。

如果未判定生成小动物,那么生成友好水中小动物的概率额外乘1/4。

\subsection{受缚哥布林}
要求:满足受缚哥布林生成条件,不是水中刷怪,刷怪面不在rockLayer之上,刷怪面距离世界底端大于210格。有1/20概率生成受缚哥布林(ID:105)。

\subsection{受缚巫师}
要求:满足受缚巫师生成条件,,不是水中刷怪,刷怪面不在rockLayer之上,刷怪面距离世界底端大于210格。有1/20概率生成受缚巫师(ID:106)。

\subsection{小动物}
要求:判定生成小动物,不是水中刷怪。丛林草皮上有1/150概率生成金蛙(ID:445),149/150概率生成青蛙(ID:361)。雪块和冰雪块上企鹅(ID:149)和蓝企鹅(ID:148)生成概率各半。在草皮或神圣草皮上的刷怪优先级:雨天>萤火虫>鸟(1)>蝴蝶>鸟(2)>兔兔和松鼠。其他情况下,如果刷怪面不在worldSurface之下,生成失败;否则生成规则与草皮上的生成规则相同,除了在刷怪面是沙块时,兔兔和松鼠会被各1/2概率生成的黑蝎子(ID:366)和蝎子(ID:367)替代。

\subsubsection{雨天}
在雨天,2/3概率生成蠕虫(ID:357),1/3概率生成步行金鱼(ID:230)。此外,有1/150的概率它们会被金蠕虫(ID:448)覆盖。

\subsubsection{萤火虫}
要求:在晚上,刷怪面不在worldSurface之下。如果刷怪面是草皮,生成萤火虫(ID:355);否则生成荧光虫(ID:358)。有1/fireFlyFriendly的概率生成。在成功生成的前提下,在刷怪面的上下左右共4格分别有1/fireFlyMultiple的概率额外生成一个。

有1/9的概率,fireFlyFriendly在1到3随机,fireFlyMultiple在3到7随机;有8/27的概率,fireFlyFriendly和fireFlyMultiple均为999999;有16/27的概率,fireFlyFriendly在2到14随机,fireFlyMultiple在6到29随机。它们的值在每天入夜时刷新。

\subsubsection{鸟(1)}
要求:在4:30到9:30,刷怪面不在worldSurface之下。有2/3概率生成某种鸟。

在确定生成某种鸟的前提下,有1/4概率生成红雀(ID:297),1/4概率生成冠蓝鸦(ID:298),1/2概率生成鸟(ID:74).此外,有1/150的概率它们会被金鸟(ID:442)覆盖。

\subsubsection{蝴蝶}
要求:白天,刷怪面不在worldSurface之下。生成某种蝴蝶的概率是1/butterflyChance。butterflyChance的值在每天入夜时刷新,有1/3概率为999999,剩下2/3概率在1到24随机。

在确定生成某种蝴蝶的前提下,有1/150概率生成金蝴蝶(ID:444),149/150概率生成蝴蝶(ID:356)。此外,在刷怪面的左右共2格分别有1/4的概率额外生成一个蝴蝶。

\subsubsection{鸟(2)}
要求:刷怪面不在worldSurface之下。有1/2概率生成某种鸟。其他刷怪概率与鸟(1)相同。

\subsubsection{兔兔和松鼠}
各种松鼠要求刷怪面不在worldSurface之下。刷怪优先级:金兔>金松鼠>史莱姆兔兔>圣诞节兔兔>派对兔兔>松鼠=红松鼠>兔兔。金兔(ID:443)概率1/150,金松鼠(ID:539)概率1/150,史莱姆兔兔(ID:303)概率2/3(要求万圣节期间),圣诞节兔兔(ID:337)概率2/3(要求圣诞节期间),派对兔兔(ID:540)概率2/3(要求派对期间),松鼠(ID:299)和红松鼠(ID:538)概率各1/6,以上所有均未成功生成的,生成兔兔(ID:46)。

\subsection{地牢}
要求:在地牢环境。刷怪优先级:地牢守卫>受缚机械师>骷髅李小龙>骷髅突击手=骷髅狙击手=骷髅特警>圣骑士=巨型诅咒骷髅头>褴褛邪教徒法师=死灵法师=魔教徒>生锈装甲骷髅=蓝装甲骷髅=地狱装甲骷髅>地牢史莱姆>尖刺球=烈焰火轮=诅咒骷髅头>暗黑法师>愤怒骷髅怪。部分敌怪只会在击败世纪之花后的困难模式出现,而且部分敌怪的生成依赖于刷怪面的背景墙种类,这两个信息下文会省略(\href{https://terraria-zh.gamepedia.com/地牢\#.E5.9B.B0.E9.9A.BE.E6.A8.A1.E5.BC.8F.E4.B8.96.E7.BA.AA.E4.B9.8B.E8.8A.B1.E5.90.8E.E7.9A.84.E5.9C.B0.E7.89.A2}{Wiki})。

如果没打过骷髅王,生成地牢守卫(ID:68)。未解救机械师,刷怪面在rockLayer之下,有1/5概率生成受缚机械师(ID:123)。骷髅李小龙(ID:287)概率1/30。骷髅突击手(ID:293)、骷髅狙击手(ID:291)、骷髅特警(ID:292)概率都是1/15。圣骑士(ID:290)概率1/35,世界中只能存在一个。巨型诅咒骷髅头(ID:289)概率1/30。褴褛邪教徒法师(ID:281/282)、死灵法师(ID:283/284)、魔教徒(ID:285/286)概率都是1/20;它们各有两个ID,生成概率均等;世界中分别只能存在一个。生锈装甲骷髅(ID:269/270/271/272)、蓝装甲骷髅(ID:273/274/275/276)、地狱装甲骷髅(ID:277/278/279/280)概率都是2/3;它们各有4个ID,生成概率均等。地牢史莱姆(ID:71)概率1/37。尖刺球(ID:70)概率1/4;要求以刷怪面为中心的边长600像素的正方形与最近的另一个尖刺球的支撑块不相交。烈焰火轮(ID:72)概率1/15。诅咒骷髅头(ID:34)概率1/9。暗黑法师(ID:32)概率1/7。

以上均未生成时,生成愤怒骷髅怪。愤怒骷髅怪一共有6个ID:294、295、296、31、-14、-13(\href{https://terraria-zh.gamepedia.com/愤怒骷髅怪}{Wiki}),它们的概率分别是1/5、1/5、1/5、6/25、1/10、3/50。

\subsection{陨石}
在陨石环境,生成流行头(ID:23)。

\subsection{撒旦军团}
撒旦军团事件在进行时,并且玩家在事件区域内,那么不刷怪。

\subsection{霜月}
要求:刷怪面不高于worldSurface,在晚上,在霜月中。刷怪优先级:礼物宝箱怪>其他。

与波数无关的是礼物宝箱怪(ID:341)。只要世界中只存在不到4个礼物宝箱怪,它就会以1/30的概率生成。

第20波,冰雪女王(ID:345)、圣诞坦克(ID:346)、常绿尖叫怪(ID:344)概率各1/3。

第19波,刷怪优先级:冰雪女王>圣诞坦克>常绿尖叫怪>雪兽。冰雪女王概率1/10,上限4个。圣诞坦克概率1/10,上限5个。常绿尖叫怪概率1/10,上限7个。这三个怪都不生成的情况下生成雪兽(ID:343)。

第18波,刷怪优先级:冰雪女王>圣诞坦克>常绿尖叫怪>其他。冰雪女王概率1/10,上限3个。圣诞坦克概率1/10,上限4个。常绿尖叫怪概率1/10,上限6个。这三个怪都不生成的情况下,胡桃夹士(ID:348)概率1/3,坎卜斯(ID:351)概率2/9,雪兽概率4/9。

第17波,刷怪优先级:冰雪女王>圣诞坦克>常绿尖叫怪>其他。冰雪女王概率1/10,上限2个。圣诞坦克概率1/10,上限3个。常绿尖叫怪概率1/10,上限5个。这三个怪都不生成的情况下,精灵直升机(ID:347)概率1/4,坎卜斯概率3/8,雪兽概率3/8。

第16波,刷怪优先级:冰雪女王>圣诞坦克>常绿尖叫怪>其他。冰雪女王概率1/10,上限2个。圣诞坦克概率1/10,上限2个。常绿尖叫怪概率1/10,上限4个。这三个怪都不生成的情况下,雪花怪(ID:352)概率1/2,雪兽概率1/2。

第15波,刷怪优先级:冰雪女王>圣诞坦克>常绿尖叫怪>其他。冰雪女王概率1/10,上限1个。圣诞坦克概率1/10,上限2个。常绿尖叫怪概率1/10,上限3个。这三个怪都不生成的情况下,精灵直升机概率1/3,雪兽概率2/3。

第14波,刷怪优先级:冰雪女王>圣诞坦克>常绿尖叫怪>其他。冰雪女王概率1/10,上限1个。圣诞坦克概率1/10,上限1个。常绿尖叫怪概率1/10,上限1个。这三个怪都不生成的情况下,雪兽概率1/3,不刷怪概率2/3。

第13波,刷怪优先级:冰雪女王>圣诞坦克>其他。冰雪女王概率1/10,上限1个。圣诞坦克概率1/10,上限1个。这两个怪都不生成的情况下,雪花怪概率1/3,雪兽概率1/9,姜饼人(ID:342)概率5/27,精灵直升机概率10/27。

第12波,刷怪优先级:冰雪女王>常绿尖叫怪>其他。冰雪女王概率1/10,上限1个。常绿尖叫怪概率1/10,上限1个。这两个怪都不生成的情况下,雪兽概率1/8,姜饼人概率7/24,僵尸精灵(ID:338/339/340)概率7/12。僵尸精灵的三个ID等概率生成。

第11波,刷怪优先级:冰雪女王>其他。冰雪女王概率1/10,上限1个。冰雪女王不生成的情况下,雪花怪概率1/6,姜饼人概率5/12,僵尸精灵概率5/12。

第10波,刷怪优先级:圣诞坦克>常绿尖叫怪>其他。圣诞坦克概率1/10,上限1个。常绿尖叫怪概率1/10,上限2个。这两个怪都不生成的情况下,坎卜斯概率1/6,胡桃夹士概率5/18,精灵直升机概率5/27,僵尸精灵概率10/27。

第9波,刷怪优先级:圣诞坦克>常绿尖叫怪>其他。圣诞坦克概率1/10,上限1个。常绿尖叫怪概率1/10,上限1个。这两个怪都不生成的情况下,胡桃夹士概率1/2,精灵直升机概率1/6,姜饼人概率1/3。

第8波,刷怪优先级:圣诞坦克>其他。圣诞坦克概率1/10,上限1个。圣诞坦克不生成的情况下,坎卜斯概率1/8,胡桃夹士概率7/24,精灵直升机概率7/36,精灵弓箭手(ID:350)概率7/18。

第7波,刷怪优先级:圣诞坦克>其他。圣诞坦克概率1/10,上限1个。圣诞坦克不生成的情况下,姜饼人概率1/3,精灵弓箭手概率1/6,僵尸精灵概率1/2。

第6波,刷怪优先级:常绿尖叫怪>其他。常绿尖叫怪概率1/10,上限2个。常绿尖叫怪不生成的情况下,精灵直升机概率1/4,胡桃夹士概率3/8,精灵弓箭手概率3/8。

第5波,刷怪优先级:常绿尖叫怪>其他。常绿尖叫怪概率1/10,上限1个。常绿尖叫怪不生成的情况下,精灵弓箭手概率1/4,胡桃夹士概率3/32,僵尸精灵概率21/32。

第4波,刷怪优先级:常绿尖叫怪>其他。常绿尖叫怪概率1/10,上限1个。常绿尖叫怪不生成的情况下,精灵弓箭手概率1/4,姜饼人概率1/4,僵尸精灵概率1/2。

第3波,胡桃夹士概率1/8,精灵弓箭手概率7/32,姜饼人概率7/32,僵尸精灵概率7/16。

第2波,僵尸弓箭手概率1/3,僵尸精灵概率2/3。

第1波,姜饼人概率1/3,僵尸精灵概率2/3。

\subsection{南瓜月}
要求:刷怪面不高于worldSurface,在晚上,在南瓜月中。

第1波,生成稻草人(ID:305/306/307/308/309/310/311/312/313/314),10个ID等概率生成。

第2波,树精(ID:326)概率1/3,稻草人概率2/3。

第3波,地狱犬(ID:329)概率1/6,树精概率5/18,稻草人概率5/9。

第4波,刷怪优先级:哀木>其他。哀木(ID:325)概率1/10,上限1个。哀木不生成的情况下,地狱犬概率1/10,树精概率9/20,稻草人概率9/20。

第5波,刷怪优先级:哀木>其他。哀木概率1/10,上限1个。哀木不生成的情况下,胡闹鬼(ID:330)概率1/8,地狱犬概率7/40,树精概率7/20,稻草人概率7/20。

第6波,刷怪优先级:哀木>其他。哀木概率1/7,上限2个。哀木不生成的情况下,胡闹鬼概率1/6,地狱犬概率5/18,树精概率5/9。

第7波,刷怪优先级:南瓜王>其他。南瓜王概率1/10,上限1个。南瓜王不生成的情况下,胡闹鬼概率1/8,地狱犬概率7/40,稻草人概率7/10。

第8波,刷怪优先级:南瓜王>其他。南瓜王概率1/10,上限1个。南瓜王不生成的情况下,胡闹鬼概率1/5,地狱犬概率4/15,树精概率8/15。

第9波,刷怪优先级:南瓜王>哀木>无头骑士>其他。南瓜王概率1/8,上限1个。哀木概率1/8,上限1个。无头骑士(ID:315)概率1/10,上限1个。这三个怪都不生成的情况下生成稻草人。

第10波,刷怪优先级:南瓜王>哀木>无头骑士>其他。南瓜王概率1/10,上限1个。哀木概率1/10,上限1个。无头骑士概率1/10,上限1个。这三个怪都不生成的情况下,胡闹鬼概率1/8,地狱犬概率7/40,树精概率7/10。

第11波,刷怪优先级:哀木>无头骑士>其他。哀木概率1/7,上限2个。无头骑士概率1/10,上限1个。这两个怪都不生成的情况下,胡闹鬼概率1/10,地狱犬概率9/70,树精概率9/35,稻草人概率18/35。如果世界中没有南瓜王,还有1/10概率额外生成一个南瓜王。

第12波,刷怪优先级:哀木>无头骑士>其他。哀木概率1/7,上限2个。无头骑士概率1/7,上限2个。这两个怪都不生成的情况下,胡闹鬼概率1/7,地狱犬概率6/35,树精概率24/35。如果世界中南瓜王数量小于2,还有1/7概率额外生成一个南瓜王。

第13波,刷怪优先级:哀木>无头骑士>其他。哀木概率1/5,上限3个。无头骑士概率1/5,上限3个。这两个怪都不生成的情况下,胡闹鬼概率1/3,地狱犬概率2/3。如果世界中南瓜王数量小于2,还有1/7概率额外生成一个南瓜王。

第14波,刷怪优先级:南瓜王>哀木>其他。南瓜王概率1/5,上限3个。哀木概率1/5,上限3个。这两个怪都不生成的情况下生成无头骑士。

第15波,南瓜王和哀木概率各为1/2。

\subsection{日食}
要求:刷怪面不在worldSurface之下,在白天,在日食中。刷怪优先级:蛾怪>眼怪>变态人>钉头>致命球>吸血鬼>死神>攀爬魔>飞人博士>屠夫>科学怪人=水月怪=弗里茨=沼泽怪。

蛾怪(ID:477)概率1/80,要求击败三个机械Boss,上限1个。眼怪(ID:251)概率1/50,上限1个。变态人(ID:466)概率1/5,要求击败世纪之花,上限1个。钉头(ID:463)概率1/20,要求击败世纪之花,上限1个。致命球(ID:467)概率1/20,要求击败世纪之花,上限2个。吸血鬼(ID:159)概率1/15。死神(ID:253)概率1/13,要求击败三个机械Boss。攀爬魔(ID:469)概率1/8。飞人博士(ID:468)概率1/7,要求击败世纪之花。屠夫(ID:460)概率1/5,要求击败世纪之花。科学怪人(ID:162)、水月怪(ID:461)、弗里茨(ID:462)、沼泽怪(ID:166)概率各1/4。

\subsection{发光蘑菇地}
要求:刷怪面为蘑菇草皮。刷怪优先级:蘑菇鱼>发光蜗牛>蘑菇僵尸=孢子僵尸=歪尾真菌=瓢虫>巨型真菌球怪>真菌球怪

蘑菇鱼(ID:256)要求困难模式,水中刷怪,一定生成。发光蜗牛(ID:360)在困难模式概率1/12,困难模式之前概率37/72。蘑菇僵尸(ID:255)和孢子僵尸(ID:254)概率各1/3,歪尾真菌(ID:257)和瓢虫(ID:258)概率各1/8。巨型真菌球怪(ID:260)概率2/3,要求困难模式。以上所有敌怪均未生成,则生成真菌球怪(ID:259)。

