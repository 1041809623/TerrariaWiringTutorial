\chapter{刷怪机制}
执行刷怪的函数为 NPC.SpawnNPC。

如果NPC.noSpawnCycle为true,那么这一帧不刷怪,把NPC.noSpawnCycle重置为false。

如果要刷怪的话,会对每个未死亡的玩家执行刷怪过程。在史莱姆雨进行时,会在其他刷怪之前插入史莱姆雨的刷怪。史莱姆雨的刷怪是额外刷怪,不会影响正常刷怪。

进行正常刷怪时,首先要判断刷怪类型、刷怪率和刷怪量,然后决定是否要刷怪,然后决定刷怪点和刷怪面,最后决定刷什么怪。

\section{史莱姆雨刷怪}

\section{刷怪类型、刷怪率和刷怪量}

\section{刷怪点和刷怪面}

\section{刷怪种类}
刷怪的优先级是四柱刷怪>太空刷怪>事件刷怪。

\subsection{四柱刷怪}
如果玩家在四柱附近(玩家中心到四柱中心的直线距离不大于4000像素),那么就会刷四柱怪。四柱刷怪的优先级是星云柱>星旋柱>星尘柱>日曜柱。

\subsubsection{星云柱}
刷怪为星云浮怪(ID:420)、吮脑怪(ID:421)、进化兽(ID:423)、预言帝(ID:424)。这四个怪的刷怪比例为1:5:3:3。星云浮怪在整个世界中的上限为2个,进化兽在整个世界中的上限为3个,预言帝在整个世界中的上限为2个。这个刷怪比例\&上限的规则是四柱的特色。举例来说,没有任何刷怪的时候,这四个怪的刷怪概率分别是1/12、5/12、1/4、1/4;如果已经刷出了两个星云浮怪,那么不会再刷星云浮怪,剩下三个怪的刷怪概率分别是5/11、3/11、3/11。

\subsubsection{星旋柱}
刷怪为漩泥怪(ID:425)、异星蜂王(ID:426)、异星黄蜂(ID:427)、星旋怪(ID:429)。刷怪比例为2:1:2:4。漩泥怪上限为3,异星蜂王上限为3,星旋怪上限为4。

\subsubsection{星尘柱}
刷怪为银河织妖(ID:402)、星细胞(ID:405)、流体入侵怪(ID:407)、闪耀炮手(ID:409)、观星怪(ID:411)。刷怪比例为1:1:1:2:3。

\subsubsection{日曜柱}
刷怪为千足蜈蚣(ID:412)、火龙怪(ID:415)、火龙怪骑士(ID:416)、火滚怪(ID:417)、流星火怪(ID:418)、火月怪(ID:419)、火龙战士(ID:518)。刷怪比例为1:1:1:1:1:1:1。千足蜈蚣上限为1,火龙怪上限为2,火龙怪骑士上限为1,火龙战士上限为2。

\subsection{太空刷怪}
优先级是火星飞船>火星探测器>飞龙>鸟妖。

如果执行的是火星入侵的刷怪,那么生成火星飞船(ID:388)。

在石巨人后,并且刷怪点到世界中心的横坐标距离大于世界宽度$\times$0.165\footnote{小世界为693格,中世界为1056格,大世界为1386格},那么有概率生成火星探测器(ID:399),这个概率与是否打过火星入侵、是否在水蜡烛区域、是否有水蜡烛buff相关(\autoref{tab5651})。水蜡烛区域和水蜡烛buff不是一回事,水蜡烛区域不包括手持水蜡烛的情况。火星探测器的上限为1。

\begin{table}[!h]
    \centering
    \begin{tabular}{cccccc}
         000&001&011&100&101&111\\\hline
         1/8&15/64&5/9&1/30&59/900&19/100 
    \end{tabular}
    \caption{二进制的第一位表示是否打过火星入侵,第二位表示是否在水蜡烛区域,第三位表示是否有水蜡烛buff。例如101表示打过火星入侵,不在水蜡烛区域内,有水蜡烛buff。}
    \label{tab5651}
\end{table}

没有水蜡烛buff的时候,飞龙(ID:87)的生成概率为1/10;有水蜡烛buff的时候生成概率为19/100。飞龙的上限为1。

如果火星飞船、火星探测器、飞龙均未生成,那么生成鸟妖。

\subsection{事件刷怪}
\subsubsection{哥布林入侵}
1/9概率生成哥布林巫士(ID:29),8/45概率生成哥布林苦力(ID:26),32/135概率生成哥布林弓箭手(ID:111),32/405概率生成哥布林盗贼(ID:27),64/405概率生成哥布林战士(ID:28)。在困难模式中,有1/30概率生成哥布林召唤师(ID:471),这会覆盖前面的生成。哥布林召唤师上限为1。

\subsubsection{雪人入侵}
1/7概率生成巴拉雪人(ID:145),2/7概率生成雪人暴徒(ID:143);4/7概率生成戳刺先生(ID:144)。

\subsubsection{海盗入侵}
1/11概率生成海盗弩手(ID:215),10/99概率生成鹦鹉(ID:252),80/693概率生成海盗神射手(ID:214),160/693概率生成私船海盗(ID:213),320/693概率生成海盗水手(ID:212)。

海盗船长(ID:216)有1/30概率生成,会覆盖前面的生成。海盗船长上限为1。

海盗船(ID:491)生成要求入侵进度超过一半,并且刷怪面的左右各20格,上方10格到40格范围内没有实体块。海盗船生成概率是1/20。海盗船上限为1。海盗船的生成会覆盖其他生成。

\subsubsection{火星入侵}
火星飞碟(ID:395)的生成分为两段判定。离入侵结束不到100分\footnote{入侵事件总分为160+40$\times$玩家数量}时,火星飞碟的概率为1/10并且会覆盖其他生成(包括第二段)。第二段中火星飞碟和其他敌怪处理方法相同。

火星飞碟上限为1,火星走妖(ID:520)上限为1。以下是第二段判定。

火星飞碟概率为1/70,鳞甲怪枪手(ID:390)和火星工程师(ID:386)概率均为9/140(火星飞碟达到上限的话,这个概率变为1/14),火星飞船(ID:388)概率为2/35,扰脑怪(ID:381)概率为4/35,激光枪手(ID:382)概率为4/35,火星走妖(ID:520)概率为1/7,灰咕噜兽(ID:385)、电击怪(ID:389)和火星军官(ID:383)概率均为1/7(火星走妖达到上限的话,这个概率为4/21)。