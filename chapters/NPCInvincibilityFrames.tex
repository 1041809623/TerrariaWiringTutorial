\chapter{NPC无敌帧机制}
\begin{center}
作者:dhuapskciao
\end{center}
\section{引入}
众所周知,泰拉瑞亚中存在无敌帧这么一个机制:当你用某些武器攻击到怪物之后,怪物会获得一段时间的无敌帧,在这一段时间内怪物“不会受到伤害”。

因此,用迷你鲨配合陨石子弹时射击某一个怪物时,连续的两发子弹只有一发能击中怪物;因此,用虎爪站撸世纪花时要特别注意不能挂雨云或者召唤小蜘蛛,否则虎爪会打不出应有的伤害。

我们知道那些能穿透敌人的子弹、魔法或剑气一般会给敌人造成无敌帧,许多召唤物也会给怪物造成无敌帧。但是在游戏过程中,细心的玩家可能会发现以上的经验会有些例外,比如说:
\begin{itemize}
\item 星尘龙的攻击尽管会造成无敌帧,但不会影响叶绿弹的输出。
\item 虽然夜明弹会穿透,但它们之间并不会互相影响输出。这个特性与陨石子弹非常不同。
\item 造成无敌帧的子弹、魔法、剑气与召唤物,它们造成无敌帧的时间并非完全一样:比如说棱镜造成的无敌帧时间明显就比鲨龙卷召唤物的碰撞造成的无敌帧时间短很多。
\end{itemize}

以及还有一些问题:
\begin{itemize}
\item 燃烧、霜火、诅咒焰等debuff会不会造成骗伤?
\item 多人联机时玩家A的攻击造成的无敌帧会不会影响玩家B的输出?
\end{itemize}

这一切都需要我们对无敌帧机制进行更深入的研究。

\section{说明}
\subsection{NPC与射弹}
NPC不仅包含电工妹和哥布林工匠等城镇NPC,还包含所有的怪物,甚至一部分怪物的攻击方式——比如地牢的黑暗法师打出的水弹,猪鲨吐出的泡泡。本文中以NPC代指敌对怪物。

\subsection{伤害途径}
玩家对NPC造成伤害的主要途径有两类:
\begin{itemize}
\item 射弹伤害NPC。
\item 用剑等近战武器挥砍。
\end{itemize}

为了以下行文方便,我们把第一类统称射弹,第二类统称近战武器。

\subsection{普通无敌帧的实现机制}
游戏中每个NPC都拥有一个计时器Immune,这是一个长度为$256$的自然数列。每帧,数列中的每个值都会下降1,直到0为止。玩家$k$的许多(具体见下文)攻击方式击中NPC时,会检测当前的NPC的Immune的第$k$个位置是否为$0$,是$0$才能伤害到敌人。于是NPC的Immune中的第$k$个数$n$表示该NPC对玩家$k$的无敌帧还会持续$n$帧。

所以游戏中不同玩家对NPC(即使是同一个NPC)造成的普通无敌帧是独立计算,互不影响的。这是游戏中最普遍的无敌帧。

当然,游戏里还有两类无敌帧,这两类无敌帧与下面会说到的两类特殊射弹绑定,所以详细的之后再说。

\section{射弹造成的无敌帧}
与无敌帧有关的射弹属性主要是射弹的穿透能力,在源码里相应的变量是Penetrate。游戏中的每个射弹都拥有自己的Penetrate值,Penetrate若为正数$k$,则表明它能击中$k$个NPC(穿透$k-1$个),如果是$-1$,则表明它能击中在自己行进路线上的所有NPC。

Penetrate为正数时,每击中一次NPC都会下降$1$,到$0$的时候这个射弹就会被销毁。

从这一方面能把射弹分成两类——穿透的(初始的Penetrate不为1)和不穿透的(初始的Penetrate=1)。注意,虽然穿透的射弹在还剩一次击中NPC的能力时penetrate=1,但还是被认为是穿透型射弹。

射弹还有两个标签:
\begin{itemize}
\item usesLocalNPCImmunity
\item usesIDStaticNPCImmunity
\end{itemize}

游戏中的700多种射弹可以根据这俩标签的有无被分成三类:
\begin{itemize}
\item 第一类是这俩标签都没有(被设定为False)的。
\item 第二类是usesLocalNPCImmunity=True的。
\item 第三类是usesIDStaticNPCImmunity=True的。
\end{itemize}

第一类射弹占了绝大多数,第二类仅仅有一小部分,第三类仅有一种:即699号射弹,食人魔掉落的关刀的本体。这三类射弹计算无敌帧、以及被无敌帧影响时的机制不尽相同。

\subsection{第一类射弹}
第一类射弹内的大多数穿透射弹在穿透NPC(击中当前NPC的时候Penetrate不是1)时,会对NPC造成10帧的无敌帧。但有一些例外,见下表:
\begin{longtable}{|c|c|c|}
\caption{特殊的第一类射弹}\\ \hline
射弹ID&射弹&Immune(帧)\\ \hline
\endfirsthead
\hline 射弹ID&射弹&Immune(帧)\\\hline
\endhead
\hline
\endfoot
632 & 棱镜激光 & 5 \\ \hline
514 & 钉枪的钉子 & 3 \\ \hline
595 & Arkhalis & 5 \\ \hline
625-628 & 星尘龙 & 6 \\ \hline
286 & 爆炸子弹 & 5 \\ \hline
443 & 电子球发射器的射弹 & 8 \\ \hline
424-426 & 陨石魔杖打出的陨石 & 5 \\ \hline
634-635 & 星云烈焰 & 5 \\ \hline
659 & 神灯烈焰 & 5 \\ \hline
246 & 毒刺发射器射弹 & 7 \\ \hline
249 & 毒刺发射器射弹的碎片 & 7 \\ \hline
190 & 机械食人鱼 & 8 \\ \hline
409 & 利刃台风 & 6 \\ \hline
407 & 鲨龙卷召唤物本体 & 20 \\ \hline
311 & 玉米糖 & 7 \\ \hline
582 & 机械师扳手 & 7
\end{longtable}

注意鲨龙卷召唤物指的是鲨龙卷本体的碰撞。至于它发射出的小鲨鱼,那东西不穿透也不造成无敌帧。

当第一类射弹碰到NPC时,如果NPC对当前玩家的Immune为0,或者射弹是不穿透型射弹,那么就能击中NPC,否则不行。第一类射弹的特点是互相影响,比如陨石子弹A对NPC造成的无敌帧会影响陨石子弹B的杀伤。当然,非穿透射弹不受影响,所以如果你用泡泡枪站撸世花,就不需要担心世花骗伤的问题;用枪械配合叶绿弹打月总时,召唤一条星尘龙出来也不会影响输出。

\subsection{第二类射弹}
\begin{longtable}{|c|c|c|c|}
\caption{第二类射弹}\\ \hline
射弹ID & 射弹 & LocalNPCImmunity & Immune\\\hline
\endfirsthead
\hline 射弹ID & 射弹 & LocalNPCImmunity & Immune\\\hline
\endhead
\hline
\endfoot
611 & 日耀喷发剑本体 & 6 & 4 \\ \hline
612 & 日耀喷发剑爆炸 & 6 & 4 \\ \hline
617 & 星云奥秘本体 & 8 & 0 \\ \hline
618 & 星云奥秘爆炸 & 8 & 0 \\ \hline
638 & 月明弹 & -1 & 0 \\ \hline
639 & 月明箭 & -1 & 0 \\ \hline
640 & 月明箭尾迹 & -1 & 0 \\ \hline
642 & 月亮传送门激光 & 10 & 0 \\ \hline
645 & 月耀 & -1 & 0 \\ \hline
656 & 禁戒套的龙卷风 & 8 & 0 \\ \hline
661 & 黑玛瑙爆破枪的黑玛瑙 & 8 & 0 \\ \hline
664,666,668 & 烈焰炮台射弹 & -1 & 0 \\ \hline
680 & 弩车炮台射弹 & -1 & 0 \\ \hline
688-690 & 闪电光环 & 3 & 0 \\ \hline
694-696 & 陷阱炮台的爆炸 & 30 & 0 \\ \hline
697 & 瞌睡章鱼本体 & 12 & 0 \\ \hline
698 & 瞌睡章鱼爆炸 & -1 & 0 \\ \hline
700 & 关刀打出的龙头 & -1 & 0 \\ \hline
704 & 无限智慧巨著的龙卷风 & -1 & 0 \\ \hline
706 & 幽灵凤凰 & 10 & 0 \\ \hline
707 & 天龙之怒旋转 & 6 & 0 \\ \hline
708 & 天龙之怒射弹 & 6 & 0 \\ \hline
710 & 空中克星箭 & -1 & 0 \\ \hline
711 & 贝特西之怒 & -1 & 0
\end{longtable}

第二类射弹被发射出去后,会拥有一个计时器localNPCImmunity,这是一个长度为200的整数列,其中的第$k$个值表明这个射弹对序号为$k$的NPC的“局部无敌时间”。这个整数列中的所有大于0的值每帧都会减少1,直到0为止。而小于0的值则不变。当第二类射弹碰到NPC时,游戏会首先检测这个射弹对这只NPC的局部无敌时间是否为0。如果不是,则无法击中;如果是,则继续检测NPC对发射这枚射弹的玩家的Immune(即普通的无敌帧)是不是0,以及这个射弹是不是非穿透射弹,如果以上至少有一个是成立的,则最终击中NPC。第二类射弹击中NPC时,绝大多数不会给NPC造成普通无敌帧,唯一的例外是喷发剑,上表里已经显示出来了。

注意,局部无敌时间是射弹的属性(而普通的无敌帧是NPC的属性),游戏中每一个射弹都拥有独立计算的局部无敌时间,加之绝大多数第二类射弹不会造成普通的无敌帧,这意味着第二类射弹基本上不会互相影响(唯一的例外是喷发剑)。所以说虽然月明弹是穿透的,但它们并不会像陨石弹那样连续发射的两发子弹只能击中一发。

此外,还可以注意到一部分第二类射弹的localNPCImmunity是-1,这意味着这个射弹无法再次击中已经击中过的NPC,而那些localNPCImmunity大于0的抛射体,则表明它在击中某个NPC之后一段时间,可以再次击中这只NPC(典型的就是星云奥秘)。

以及由于第二类射弹碰到NPC时仍然会检测NPC的Immune,所以普通无敌帧还是会影响第二类射弹的输出。简单来说,虽然第二类射弹基本不会互相影响,但它们会受到第一类射弹的影响。

\subsection{第三类射弹}
只有那个ID为699的大关刀的本体。

与那个东西有关的是perIDStaticNPCImmunity,这个变量是个714(游戏内射弹总种类数)$\times$200的整数矩阵,这是个游戏的内置变量,不属于任何一个NPC或者射弹。关刀碰到序号为$k$的NPC时,会检测矩阵第700行,第$k+1$列的数字是否小于等于GameUpdateCount,如果是则继续检测NPC对发射这枚射弹的玩家的Immune是不是0,以及这个射弹是不是非穿透抛射体,如果以上至少有一个是成立的,则关刀最终击中NPC。

关刀击中NPC后,也会在矩阵的相应位置赋值GameUpdateCount+36。这个大矩阵内的数值本身是不会下跌的(不像Immune和localNPCImmunity),但因为GameUpdateCount会上涨(速率应该也是每帧1),所以起到的倒计时效果一样。

所以关刀的特性和第二类射弹差不多,但有一点不一样:不同玩家的关刀会互相影响输出。

\section{近战武器造成的无敌帧}
游戏每帧都会检测每个NPC对某个玩家的Immune是不是0,以及玩家的attackCD是不是0,如果以上都成立,并且武器碰到了NPC,则对NPC造成伤害。玩家的attackCD=玩家itemAnimationMax/3。而itemAnimationMax是对应武器的useAnimation*玩家的meleespeed(为实际近战攻速的倒数),useAnimation是武器的自身属性。所以实战中,挥舞一次剑能打到的NPC数量是有限的。并且挥舞武器受普通无敌帧的影响。

此外,当挥舞武器给NPC造成伤害时,会给NPC对当前玩家的一个比较短的普通无敌帧,持续时间是当前玩家的itemAnimation。这个数值在你刚开始挥舞武器的时候与itemAnimationMax一样,然后下降,如果没理解错的话,当你挥舞完成的时候,这个值会降到0.这个无敌帧的存在应当只是为了保证玩家每一次挥砍,对同一个NPC只能造成一次伤害。

\section{零散内容}
以下是一些零散内容:
\begin{itemize}
\item 着火、诅咒焰以至于星尘细胞吸附、破晓等烧血debuff通过NPC.UpdateNPC\_Buff\allowbreak ApplyDOTs函数直接作用于NPC生命值,不受无敌帧影响也不造成无敌帧。
\item 部分射弹在击中NPC后属性会改变——主要是那些击中目标后爆炸的射弹比如爆炸子弹、月耀等。改变的属性主要是射弹的大小,不过有时候Penetrate属性也会变,这造成了一些有趣的现象:爆炸子弹自身Penetrate是1,但击中NPC后Penetrate变成-1,长宽变成80像素然后再造成一下伤害。于是爆炸子弹本体伤害不受无敌帧影响,但是爆炸的范围伤害会受到无敌帧影响。同样的现象在星云烈焰上也能观察到(星云烈焰的爆炸是范围伤害,范围是50*50像素)。
\item 岩浆对NPC造成30帧的普通无敌帧,用NPC的Immune列表的第256个数储存并计算,游戏中能存在的玩家总数为255因此不会影响玩家的输出。
\item 陷阱也会对NPC造成无敌帧,并且会被算到玩家头上,因此会影响玩家的输出。
\end{itemize}