\chapter{电路优化}

在众多泰拉瑞亚视频中,电路视频最不受欢迎,除了晦涩的逻辑以外,一个重要的原因就是很少有电路照顾到审美。很多人做出的演示电路,仅仅可以满足功能上的要求(有些甚至功能都满足不了),而电线和电路部件连接混乱、丑陋,不仅可读性差,更会直接劝退一些对电路有兴趣的人。在这一章中,我们提出一些评价电路的标准。在后面章节中的各个例子基本上都遵循这个理念建造电路。

首先,一个电路当然需要能完成一定的功能,即需要有正确性。一个电路具有正确性,需要满足如下要求:

\begin{itemize}
	\item 触发要求:对所有需要触发的情形都能正确触发,对所有不需要触发的情形都能避免意外触发。
	\item 运行要求:能胜任所需的工作,并且在任何情况下不会意外中止运行。
	\item 适用范围:对所有可能出现的输入都能正确处理。
\end{itemize}

这里需要考虑的意外情况非常多,例如退出游戏重进是否会干扰电路?会不会刷怪意外触发压力板,或者使用武器时意外触发青绿压力垫板?在面对某些输入时是否会意外爆门?如果不能满足这些要求,那么需要将电路失效的所有情形列举出来放在醒目位置,以供用户参阅。只要用户按照“说明书”使用该电路不会出bug,也可以说这个电路有正确性。

尽管编写详尽的说明书可以保证正确性,我们当然希望这个说明书内容越短越好,即电路要能智能处理尽可能多的意外情况,这就是鲁棒性(robustness,专业名词,请参阅有关资料)。一个电路的鲁棒性越差,用户就越需要小心操作以避免bug。

正确性和鲁棒性都是功能上的要求,接下来就是外观上的要求。

在功能完全相同的情况下,显然占用面积越小越好。占用面积小不光可以为其他电路和建筑留出空间,有时更可以使电路更紧凑,增强可读性。占用面积一般包括两个参数,即宽度和高度。宽度和高度往往不能同时优化,这时就需要视电路建造的实际需求。在控制住其中一个参数的前提下尽可能减少另一个参数。

电路的可读性越强,电路就越易于维护,即在出现意外的时候更容易修复。同时,可读性强的电路也更利于他人学习。能看懂电路的人越多,电路的价值就越大。要增强电路的可读性,需要尽可能减少电线的重叠,减少爆门的使用,并使具有相同功能的模块尽可能有完全相同的放置和接线,尽管这可能消耗更多电线。美观性和可读性是正相关的,因此不单独列举出美观性。

综上所述,优化电路时需要考虑的方面有:

\begin{itemize}
\item 正确性
\item 鲁棒性
\item 占用面积
\item 可读性
\end{itemize}

很多时候上面的一些性质会冲突,例如使用分线盒会减少占用面积,同时降低可读性。为了实现强的鲁棒性,需要增加很多处理电路,从而增加占用面积。在这些情况下,需要设计者根据实际需求进行适当的取舍,或者提出多种方案。